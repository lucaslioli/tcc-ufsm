%%%% Arquivo base para o documento - ver. 1.00 (24/02/2016)
% % % % % % % % % % % % % % % % 
% % % % % % % % % % % % % % % % 
%%%%%% MDT UFSM 2015 %%%%%%%%%%
% % % % % % % % % % % % % % % % 
% % % % % % % % % % % % % % % % 
% % %  OPÇÕES DE COMPILAÇÃO  %%%%%%%%%%%


% % % % % PAGINAÇÃO
% % % PAGINAÇÃO SIMPLES (FRENTE): PARA TRABALHOS COM MENOS DE 100 PAGINAS
\documentclass[oneside,openright,12pt]{ufsm_2015} %%%%% OPÇÃO PADRÃO -> PAGINAÇÃO SIMPLES. PARA TRABALHOS COM MAIS DE 100 PAGINAS COMENTE ESTA LINHA E DESCOMENTE A LINHA 
% % % % % % % % % % % % % % % % % % % % % % % % % % % % % % % % % % % % % % %
% PAGINAÇÃO DUPLA (FRENTE E VERSO): PARA TRABALHOS COM MAIS DE 100 PAGINAS
% \documentclass[twoside,openright,12pt]{ufsm_2015}  %%%% PARA MAIS DE 100 PAGINAS DESCOMENTAR
% % % % % % % % % % % % % % % % % % % % % % % % % % % % % % % % % % % % %

% % % % % % % % % % % % % % % % % % % % % % % % % % % % % % % % % % % % %
% % % % % % % % % % % % % % % % % % % % % % % % % % % % % % % % % % % % %
%%%%%%%%% DEFINIÇÃO PADRÃO DE PACOTES -- ALTERE POR SUA CONTA E RISCO
% % % % % % % % % % % % % % % % % % % % % % % % % % % % % % % % % % % % %

\usepackage{amsmath}
\usepackage{enumerate}
\usepackage{amssymb}
\usepackage{graphicx}
\usepackage{epsf,amsfonts}
\usepackage{amsfonts}
\usepackage{epstopdf}
\usepackage{float}
% % % %  PACOTE DE CODIFICAÇÃO - PADRÃO = UTF8
\usepackage[utf8]{inputenc}  %utf8
% \usepackage[latin1]{inputenc}   % europeu
% % % % % % % % % % % 
\usepackage[brazil]{babel}
\usepackage[T1]{fontenc}
\usepackage{indentfirst}
\usepackage{textcomp}
\usepackage{setspace}
\usepackage{picinpar}
\usepackage{ifthen}
\usepackage{path}
\usepackage{scalefnt}
\usepackage{tocloft}
\usepackage[overload]{textcase}

\usepackage{listings}

% % % % % % % % % % % % % % % % % % % % % % % % % % % % % % % % % % % % % % % % % % % 
% % % % % % % % FIM DA DEFINIÇÃO PADRÃO DE PACOTES  % % % % % % % % 
% % % % % % % % % % % % % % % % % % % % % % % % % % % % % % % % % % % % % % % % % % % 


% % % % % % % % PACOTES PESSOAIS % % % % % % % %  


% % % % % %  DEFINIÇÕES PESSOAIS  


% % % % % % % % % % % % % % % % % % % % % % % % % % % % % % % % % % % % % % % % % % % 


% % % % % % % % % % % % % % % % % % % % % % % % % % % % % % % % % 
% % % % % % % % % % % % DADOS DO TRABALHO % % % % % % % % % % % % 
% % % % % % % % % % % % % % % % % % % % % % % % % % % % % % % % % 

% % % % % % % % % % INFORMAÇÕES INSTITUCIONAIS % % % % % % % % % % 
% % CENTRO DE ENSINO DA UFSM
\centroensino{Centro de Tecnologia}  %%% NOME POR EXTENSO
\centroensinosigla{CT}  %%% SIGLA

% % CURSO DA UFSM
\nivelensino{Graduação}  %%%%%%% NÍVEL DE ENSINO 
\curso{Sistemas de Informação}   %%%%% NOME POR EXTENSO
\ppg{PPGALGO}   %%%%%% SIGLAregister_error_handler
\statuscurso{Curso}  %%%% STATUS= {Programa} ou {Curso}


% % % % % % % % % % INFORMAÇÕES DO AUTOR % % % % % % % % % % 
\author{Lucas Lima de Oliveira}   %%%%% AUTOR DO TRABALHO
\sexo{M} %%%% SEXO DO AUTOR -> M=masculino   F=feminino (IMPORTANTE PARA AJUSTAR PAGINAS PRE-TEXTUAIS)
\grauensino{Graduação}    %%%%%%%% GRAU DE ENSINO A SER CONCLUÍDO
\grauobtido{Bacharel}    %%%%% TITULO OBTIDO
\email{loliveira@inf.ufsm.com.br}   %%%% E-MAIL PARA CATALOGRÁFICA (COPYRIGHT) - OBRIGATÓRIO
% \endereco{Nome da Rua, n. 999} %%%% ENDEREÇO PARA CATALOGRÁFICA (COPYRIGHT) - OPCIONAL
% \fone{11 2222 3333} %%%% TELEFONE PARA CATALOGRÁFICA (COPYRIGHT) FORMATO {11 2222 3333} - OPCIONAL
% \fax{11 2222 3333}  %%%% FAX PARA CATALOGRÁFICA (COPYRIGHT) FORMATO {11 2222 3333} - OPCIONAL


% % % % % % % % % % INFORMAÇÕES DA BANCA % % % % % % % % % % 
% OBSERVAÇÕES: O CAMPO ORIENTADOR É OBRIGATÓRIO E NÃO DEVE SER COMENTADO
% % % % % %    OS DEMAIS MEMBROS DA BANCA (COORIENTADOR E DEMAIS PROFESSORES) QUANDO COMENTADOS NÃO APARECEM NA FOLHA DE APROVAÇÃO (O LAYOUT DA FOLHA DE APROVAÇÃO ESTA PREPARADO PARA O ORIENTADOR E ATE MAIS 4 MEMBROS NA BANCA

\orientador{Sérgio Luís Sardi Mergen}{Dr}{UFSM}{M}{P}  %%%INFORMAÇÕES SOBRE ORIENTADOR: OS CAMPOS SÃO:{NOME}{SIGLA DA TITULAÇÃO}{SIGLA DA INSTITUIÇÃO DE ORIGEM}{SEXO} M=masculino F=feminino {PARTE DA BANCA?} P=presidente M=Membro  N=Não faz parte

% \coorientador{Coorientador}{Dr}{AAAA}{M}{M} %%%INFORMAÇÕES SOBRE CO-ORIENTADOR: OS CAMPOS SÃO:{NOME}{SIGLA DA TITULAÇÃO}{SIGLA DA INSTITUIÇÃO DE ORIGEM}{SEXO} M=masculino   F=feminino {PARTE DA BANCA?} P=presidente  M=Membro  N=Não faz parte

\bancaum{Banca Um}{Dr}{AAAA}{M}{M}  %%%INFORMAÇÕES SOBRE PRIMEIRO NOME DA BANCA: OS CAMPOS SÃO:{NOME}{SIGLA DA TITULAÇÃO}{SIGLA DA INSTITUIÇÃO DE ORIGEM}{SEXO} M=masculino   F=feminino {PARTE DA BANCA?} P=presidente  M=Membro  N=Não faz parte

\bancadois{Banca Dois}{Dr}{BBBB}  %%%INFORMAÇÕES SOBRE SEGUNDO NOME DA BANCA: OS CAMPOS SÃO:{NOME}{SIGLA DA TITULAÇÃO}{SIGLA DA INSTITUIÇÃO DE ORIGEM}

% \bancatres{Banca Três}{Dra}{CCCC} %%%INFORMAÇÕES SOBRE TERCEIRO NOME DA BANCA: OS CAMPOS SÃO:{NOME}{SIGLA DA TITULAÇÃO}{SIGLA DA INSTITUIÇÃO DE ORIGEM}

% \bancaquatro{Banca Quatro}{Dr}{DDDD} %%%INFORMAÇÕES SOBRE QUARTO NOME DA BANCA: OS CAMPOS SÃO:{NOME}{SIGLA DA TITULAÇÃO}{SIGLA DA INSTITUIÇÃO DE ORIGEM}

% \bancacinco{Banca Cinco}{Dra}{EEEE} %%%INFORMAÇÕES SOBRE QUARTO NOME DA BANCA: OS CAMPOS SÃO:{NOME}{SIGLA DA TITULAÇÃO}{SIGLA DA INSTITUIÇÃO DE ORIGEM}



% % % % % % % % % % INFORMAÇÕES SOBRE O TRABALHO % % % % % % % % % %
% % % %  TITULO DO TRABALHO
\titulo{Utilização de Algoritmos de Aprendizado de Máquina para Prever a Popularidade de Tuítes} %% NÃO EH NECESSÁRIO CAPITALIZAR

% % % %  TITULO DO TRABALHO EM INGLÊS
\englishtitle{Use of Machine Learning Algorithms to Predict Tweets Popularity}  %% NÃO EH NECESSÁRIO CAPITALIZAR

% % % ÁREA DE CONCENTRAÇÃO DO TRABALHO (CNPQ)
\areaconcentracao{Área de concentração do CNPq}
% % % TIPO DE TRABALHO - MANTER APENAS UMA LINHA DESCOMENTADA
% \tccg  %% Tese de <nível de ensino>
% \qualificacao %% Exame de Qualificação de <nível de ensino>
% \dissertacao %% Dissertação de <nível de ensino>
% \monografia %% Monografia
% \monografiag  %% Monografia (não exibe área de concentração)
% \tf  %% Trabalho Final de <nível de ensino>
% \tfg  %% Trabalho Final de Graduação (não exibe área de concentração)
% \tcc  %% Trabalho de Conclusão de Curso
\tccg  %% Trabalho de Conclusão de Curso (não exibe área de concentração)
% \relatorio  %% Relatório de Estágio (não exibe área de concentração)
% \generico   %%% Alternativa para aqueles cursos que não recebem o titulo de bacharel ou licenciado. Ex: engenharia, arquitetura, etc... Os campos abaixo também devem ser preenchidos
%     \tipogenerico{Tipo de trabalho em português}
%     \tipogenericoen{Tipo de trabalho em inglês}
%     \concordagenerico{o}
%     \graugenerico{Engenheiro Eletricista}
% % % DATA DA DEFESA 
\data{25}{12}{2018} %% FORMATO {DD}{MM}{AAAA}



% % % % %  ALGUMAS ENTRADAS PRE-TEXTUAIS
% % % % CASO NÃO QUEIRA UTILIZA-LAS COMENTE A LINHA DE COMANDO
% % % EPIGRAFE
\epigrafe{O livro é uma criatura frágil, ele sofre o desgaste do tempo, ele teme os roedores, os elementos e mãos desajeitadas. Então o livreiro proteje os livros não apenas da humanidade, mas também da natureza e devota sua vida a uma guerra contra as forças do esquecimento.}{Umberto Eco} %ESTRUTURA DE CAMPOS -> {Texto}{Autor}
% % % DEDICATÓRIA
\dedicatoria{Ao Rei da Espanha!}
% % % %  AGRADECIMENTOS
\agradecimentos{A mim!}

% % % % %  RESUMO E PALAVRAS CHAVE DO RESUMO - OBRIGATÓRIO PARA MDT-UFSM
\resumo{
Resumo aqui.
}
\palavrachave{Palavra Chave 1. Palavra 2. Palavra 3. (...)}
% "... deverão constar, no mínimo, três palavras-chave, iniciadas em
% letras maiúsculas, cada termo separado dos demais por ponto, e
% finalizadas também por ponto." MDT 2012

% % % % %  ABSTRACT E PALAVRAS CHAVE DO RESUMO - OBRIGATÓRIO PARA MDT-UFSM
\abstract{
Abstract here.
}
\keywords{Keyword 1. Keyword 2. Keyword 3. (...)}


% % %  ATIVAÇÃO DE LISTAS E PAGINAS ESPECIAIS
% % %  PARA QUE NÃO APAREÇAM NO TEXTO DESCOMENTE A LINHA ABAIXO -> POR PADRÃO TODAS ESTÃO ATIVIDADES

% % LISTA DE FIGURAS 
% \semfiguras   %%(QUANDO ATIVADA NÃO EXIBE A LISTA)
% % LISTA DE GRÁFICOS 
% \semgraficos   %%(QUANDO ATIVADA NÃO EXIBE A LISTA)
% % LISTA DE ILUSTRAÇÕES 
\semilustracoes  %%(QUANDO ATIVADA NÃO EXIBE A LISTA)
% % LISTA DE TABELAS 
% \semtabelas   %%(QUANDO ATIVADA NÃO EXIBE A LISTA)
% % LISTA DE QUADROS 
% \semquadros   %%(QUANDO ATIVADA NÃO EXIBE A LISTA)
% % LISTA DE APÊNDICES 
% \semapendices  %%(QUANDO ATIVADA NÃO EXIBE A LISTA)
% % LISTA DE ANEXOS 
% \semanexos   %%(QUANDO ATIVADA NÃO EXIBE A LISTA)



% % % %  LISTA DE ABREVIATURAS E SIGLAS
%%%%%%%% OBS: O espaço entre colchetes \item[] e um ambiente matemático
%%%%%%%% para não utilizar comente as linhas abaixo.
\siglamax{HTTP} %%%% coloque aqui a maior sigla (indentação)
\listadeabreviaturasesiglas{
\item[API] \textit{Application Programming Interface}
\item[HTTP] \textit{Hypertext Transfer Protocol}
\item[JSON]    \textit{JavaScript Object Notation}
}

% % % %  LISTA DE SÍMBOLOS
%%%%%%%% OBS: O espaço entre colchetes \item[] e um ambiente matemático
%%%%%%%% para não utilizar comente as linhas abaixo.
% \simbolomax{(Re)2} %%%% coloque aqui o maior simbolo (indentação)
% \listadesimbolos{
% \item[u_*]	Escala de velocidade de fricção	
% \item[w_*]	Escala de velocidade convectiva
% \item[(Re)^2]	Maior simbolo da lista
% }


% % FICHA CATALOGRÁFICA
% \semcatalografica  %%%%  (QUANDO ATIVADA NÃO EXIBE A FICHA CATALOGRÁFICA NECESSITA DO ARQUIVO DA FICHA: ficha_catalografica.pdf

% % % A FICHA CATALOGRÁFICA FORNECIDA PELA UFSM EH UM PDF DO TAMANHO A4
% % % OS COMANDOS ABAIXO DEFINEM AS MARGENS PARA CORTAR A FICHA FORNECIDA E COLOCA-LA COMO UMA FIGURA NO DOCUMENTO LATEX
\margemesquerda{4}   %%%% CORTE DE MARGEM ESQUERDA EM CM
\margemdireita{1.5}   %%%% CORTE DE MARGEM DIREITA EM CM
\margemsuperior{17}  %%%% CORTE DE MARGEM SUPERIOR EM CM
\margeminferior{3} %%%% CORTE DE MARGEM INFERIOR EM CM
% % %  DICA: IMPRIMA UMA COPIA DA FICHA CATALOGRÁFICA E FACA A MEDIDA DAS MARGENS!


% % FOLHA DE ERRADA (versão rudimentar...pode ser aprimorado)
% % para não utilizar comente as linhas abaixo.
% % deve ser preenchida como um ambiente tabular de quatro colunas:
% % pagina & linha & onde se lê & leia-a se \\
%\errata{
%10   &    10    & errado   & certo \\
%\hline
%12    &    5     & errado com um texto mais longo & certo agora com um texto mais longo\\
%\hline
%13   &    3    & $x^2$   & $2x$\\
%}
% % % % % % % % % % % % % % % % % % % % % % % % % % % % % % % % % % % % % % % % % % % % % % 


% % % % % % % % % % % % % % % % % % % % % % % % % % % % % % % % % % % % % % 
% % % % % % % % % % % %  OPÇÕES DE FORMATAÇÃO % % % % % % % % % % % % % % %
% % % % % % % % % % % % % % % % % % % % % % % % % % % % % % % % % % % % % % 
% % % CAPITULO: por padrão alinhado a esquerda. Para ativar alinhamento centralizado descomente o comando abaixo

%\centralizado  %%%% <<< centraliza todos os capítulos

% % % % % % % % % % % % % % % % % % % % % % % % % % % % % % % % % % % % % %
% % % FONTES: descomente uma das opções. caso nenhuma seja ativada a classe usara a fonte padrão do latex

%% helvetica
% \usepackage[scaled]{helvet}
% \renewcommand*\familydefault{\sfdefault}

%% arial
\renewcommand{\rmdefault}{phv} % Arial
\renewcommand{\sfdefault}{phv} % Arial

%%times
% \usepackage{mathptmx}

% % % % % % % % % % % % % % % % % % % % % % % % % % % % % % % % % % % % % % 
% % % % % % % % % % % % % % % % % % % % % % % % % % % % % % % % % % % % % % 
% % % % % % % % % % % % % % % % % % % % % % % % % % % % % % % % % % % % % % 
% % % % % % % % % % % % % % % % % % % % % % % % % % % % % % % % % % % % % % 


% % % % % % % % % % % % % % % % % % % % % % % % % % % % % % % % % % % % % % 
% % % % % % % % % % % % % % % % % % % % % % % % % % % % % % % % % % % % % % 
% % % % % % % % % % % %  INICIO DO DOCUMENTO  % % % % % % % % % % % % % % %
% % % % % % % % % % % % % % % % % % % % % % % % % % % % % % % % % % % % % % 
% % % % % % % % % % % % % % % % % % % % % % % % % % % % % % % % % % % % % %


\begin{document}


% % % % % % % % % % % % % % % % % % % % % % % % % % % % % % % % % % % % % % 
\pretextual  %%%% GERA AS PAGINAS PRE-TEXTUAIS   
% % % % % % % % % % % % % % % % % % % % % % % % % % % % % % % % % % % % % % 

% % % % % % % % % % % % % % % % % % % % % % % % % % % % % % % % % % % % % % 
% % % % % CORPO DO TRABALHO - INCLUA OS SEUS TEXTOS AQUI
% % % % % SUGESTÃO -> UTILIZE ARQUIVOS EXTERNOS A PARTIR DO COMANDO \input


% % % % % % % % % % % % % % % % % % % % % % % % % % % % % % % % % % % % % % 
% % % % % % % % % % INICIO DAS PAGINAS TEXTUAIS % % % % % % % % % % % % % % 
% % % % % % % % % % % % % % % % % % % % % % % % % % % % % % % % % % % % % % 


% % % % % % % % % % % % % % % % % % % % % % % % % % % % % % % % % % % % % % 
% % % % % % % % % % % % % INTRODUÇÃO % % % % % % % % % % % % % % % % % % % % 
% % % % % % % % % % % % % % % % % % % % % % % % % % % % % % % % % % % % % % 
\introducao{

\par Com a grande popularização dos chamados influenciadores digitais, é notável o crescimento das mídias sociais como meios de comunicação e divulgação de conteúdos. Neste cenário, onde o número de seguidores determina a sua influência, torna-se muito importante que essas personalidades compreendam seu público, pois conteúdos direcionados refletem diretamente no alcance de suas publicações. Dentre as redes sociais mais utilizadas atualmente, o Twitter é um meio de veiculação de mensagens que se destaca, por sua simplicidade e objetividade. Embora não tenha o mesmo destaque que outras plataformas, como o Facebook ou o Instagram, o Twitter conta com cerca de 335 milhões de usuários ativos, segundo Statista~\footnote{\textit{Statista: }\texttt{https://www.statista.com/topics/737/twitter/}}, e em média 500 milhões de tuítes que são publicados diariamente, segundo \textit{Internet Live Stats}~\footnote{\textit{Internet Live Stats: }\texttt{http://www.internetlivestats.com/twitter-statistics/}}, o que faz dessa rede uma fonte de dados muito poderosa.

\par Uma das preocupações de usuários do Twitter é alavancar sua popularidade, através do aumento no número de seguidores. Essa preocupação é fundamental para empresas e personalidades públicas que utilizam suas imagens para fins monetários. Nesses casos, o uso das redes sociais deve ser planejado e monitorado. Quando isso é realizado da maneira correta, a marca e/ou a pessoa ficam muito mais próximos de seus fãs e seguidores, o que consequentemente, faz sua popularidade e influência aumentar. Como mencionado em \cite{artigo:oliveira:18}, um dos indicadores capaz de medir a influência de um usuário no Twitter, é a quantidade de retuítes que suas mensagens recebem, presente inclusive na fórmula para o cálculo de engajamento, o qual será explanado posteriormente. Considerando esse fator, pode-se afirmar empiricamente que o aumento na quantidade de retuítes leva a um aumento na quantidade de seguidores, devido a propagação exponencial daquele conteúdo.

\par Tendo em vista o interesse dos usuários em aumentar o alcance de suas postagens, poder identificar os fatores que têm maior influência sobre a popularidade de suas mensagens pode ser uma grande vantagem ao tentar aumentar o engajamento por parte de seus seguidores. Ser capaz de prever/estimar a popularidade que um tuíte poderá obter, baseando-se nas características presentes no corpo de sua mensagem, permite a realização de diferentes análises a cerca o conteúdo disseminado por aquela conta, o que pode trazer muitos benefícios aos usuários com relativa influência nessa rede social.

\par Como afirma \cite{ieee:suh:10}, a propagação de um tuíte está diretamente ligada ao conteúdo e valor informativo contido nele. Nesse sentido, os autores avaliaram um conjunto de características extraídas das mensagens. Os resultados mostraram que a utilização de \textit{hashtags} e URLs são fatores muito significativos e que ajudam a impulsionar uma publicação. Apesar de ser um resultado relevante, o trabalho não realizou uma análise exaustiva das características que podem ser extraídas das mensagens.

\par Dentro deste contexto, o objetivo deste trabalho é monitorar e extrair tuítes de determinadas contas do Twitter, a fim de elaborar um modelo, utilizando algoritmos de aprendizado de máquina, para realizar a predição e classificação da popularidade de tuítes com base em suas características. Devido ao grande volume de dados, faz-se necessário automatizar o processo de análise e classificação dos dados, para isso, serão estudados e testados algoritmos já consolidados, como Naive Bayes, J48 e LTM, contando inclusive com o auxílio de ferramentas já reconhecidas como Weka, para a aplicação e análise dos resultados destes algoritmos. Como entrada para estes algoritmos, serão utilizados dados provenientes do pré-processamento dos tuítes coletados, sendo consideradas três características que não foram contempladas pelo estudo de \cite{ieee:suh:10}: o tamanho em caracteres, o sentimento (que mede a emoção transmitida) e a banalidade (que mede a relevância da mensagem). Para fins de comparação, a presença de \textit{hashtags} e URLs também foi avaliada.

\par Este trabalho está estruturado nas seguintes seções. O capítulo \ref{sec:fund-teorica} apresenta a fundamentação teórica, abordando conceitos e algoritmos de aprendizado de máquina. O capítulo \ref{sec:trab-relacionados} apresenta os trabalhos relacionados. O capítulo \ref{sec:proposta} apresenta a definição dos atributo e a arquitetura de extração de tuítes usada, que realiza desde a coleta até a preparação dos dados para análise. O capítulo \ref{sec:experimentos} apresenta as análises realizadas a partir dos dados coletados juntamente com os algoritmos de aprendizado de máquina estudados. O capítulo \ref{sec:conclusao} apresenta as considerações finais a cerca do trabalho realizado.

}
% % % % % % % % % % % % % % % % % % % % % % % % % % % % % % % % % % % % % % 
\geraintro  %%%% GERA INTRODUÇÃO   % % % % % % % % % % % % % % % % % % % % % 
% % % % % % % % % % % % % % % % % % % % % % % % % % % % % % % % % % % % % % 
% % % % % % % % % % % % % % % % % % % % % % % % % % % % % % % % % % % % % % 

% % % % % % % % % % % % % % % % % % % % % % % % % % % % % % % % % % % % % % 
% % % % % % % % % % % % FUNDAMENTAÇÃO TEÓRICA  % % % % % % % % % % % % % % % 
% % % % % % % % % % % % % % % % % % % % % % % % % % % % % % % % % % % % % % 

\chapter{Fundamentação Teórica}
\label{sec:fund-teorica}

\par Neste capítulo serão apresentados os conceitos relacionados ao aprendizado de máquina, na seção \ref{sec:aprend-maquina}, definindo as diferenças entre a aprendizagem supervisionada e a não supervisionada. Em seguida, na seção \ref{sec:alg-aprend-maquina-sup}, são apresentados alguns dos principais algoritmos do segmento supervisionado, os quais também foram utilizados na realização de experimentos no decorrer deste trabalho, sendo estes o Naive Bayes, Árvores de Decisão e Algoritmos Genéticos.

\section{Aprendizado de Máquina}
\label{sec:aprend-maquina}

\par Entende-se como sistemas inteligentes, aqueles que são capazes de processar dados de entrada e ajustar padrões internos a fim de otimizar seus resultados de saída, de acordo com os objetivos esperados para aquele algoritmo. Dentro deste contexto, o aprendizado de máquina foca no treinamento desses algoritmos para melhorar seu desempenho. Esse processo está ligado com a redução de dimensionalidade, classificação e associação dos dados e previsão de comportamentos.

\par Algoritmos de aprendizado de máquina (ou \textit{machine learning} em inglês) dividem-se em dois segmentos, aqueles que necessitam de uma supervisão para melhorar seus resultados e aqueles fazem esse processo de maneira independente. Nesta seção serão apresentados esses dois tipos de algoritmos, especificando suas características e diferenças.

\subsection{Aprendizado de Máquina Supervisionado}
\label{sec:aprend-maquina-sup}

\par A aprendizagem supervisionada realiza o treinamento dos algoritmos com dados para os quais suas respostas já sejam conhecidas. Ou seja, dependem sempre da entrada de um padrão de valores e da comparação das respostas do sistema com aquelas consideradas corretas. Conforme o algoritmo é treinado seus padrões vão sendo ajustados a fim de diminuir o erro e otimizar as respostas. 

\par Os problemas solucionados através da aprendizagem supervisionada são divididos em problemas de regressão e classificação de dados, como ilustra a Figura \ref{fig:aprend-sup}, o algoritmo recebe as entradas já classificadas para realizar o treinamento e, a cada iteração, ajusta seus parâmetros para de obter a melhor saída, podendo ser, por exemplo, minimizar o erro, maximizar a precisão ou a acurácia. Frequentemente, após a etapa de treinamento, é realizada uma etapa de validação, passando ao algoritmo entradas sem classificação, dessa forma seu desempenho pode ser realmente avaliado e, se necessário, o treinamento pode ser realizado novamente com novos ajustes em seus parâmetros.

\begin{figure}[ht]
     \caption{Exemplo de Problemas de Classificação e Regressão.}
\centering
\includegraphics[width=0.6\textwidth]{figuras/aprend-sup.png}
\vspace{\baselineskip} %%% linha em branco para atender a norma
    \fonte{\citeonline{site:medium:pedro}.}
\label{fig:aprend-sup}
\end{figure}

\subsection{Aprendizado de Máquina Não Supervisionado}
\label{sec:aprend-maquina-nao-sup}

\par No caso dos algoritmos de aprendizagem não supervisionada, ao contrário do segmento apresentado na subseção anterior, estes recebem os dados sem nenhum classificação prévia, impossibilitando o aferimento das classes de cada entrada. Consequentemente, conforme os dados vão sendo recebidos, o próprio algoritmo é responsável por identificar as relações e padrões presentes nos dados, o que por si só pode ser considerado um objetivo a ser alcançado. A aprendizagem não supervisionada não prevê soluções específicas para realizar o treinamento e validação dos resultados, o que torna a implementação deste tipo de algoritmo mais custosa.

\section{Algoritmos de aprendizado de máquina supervisionado}
\label{sec:alg-aprend-maquina-sup}

\par Dentro do escopo deste trabalho, que tem como um dos objetivos realizar a predição da popularidade de tuítes, requere-se a utilização da aprendizagem de máquina supervisionada, pois os resultados esperados estão diretamente ligados a classificação dos dados. Como mencionado, nesta seção serão abordados alguns dos principais algoritmos que se encaixam neste segmento e que serão utilizados no decorrer deste trabalho, apresentado suas características, funcionamento, vantagens e desvantagens na sua utilização.

\subsection{Naive Bayes}
\label{sec:naive-bayes}

\par A técnica Naive Bayes pode ser considerada como uma das mais populares para classificação de dados utilizando aprendizado de máquina. O algoritmo utiliza de métodos probabilísticos, baseados na Teoria Bayesiana, criada por Thomas Bayes no século XVIII. Para compreender melhor o funcionamento dessa técnica, é importante entender também alguns conceitos acerca do teorema do qual ela teve origem.

\par O Teorema de Bayes é uma formula simples, definida na Equação \ref{eq:teorema-bayes}, que vem da regra do produto de probabilidades, assumindo que \textit{prob(D|H) = prob(H|D)}, sendo D os dados observados e H a hipótese a ser validada. Definindo as parte do teorema, do lado esquerdo, \textit{prob(H|D)} é chamada de probabilidade posterior da hipótese após a realização do experimento; do lado direito, \textit{prob(D|H)} chamada função de verossimilhança, é a distribuição de probabilidade dos dados, a qual multiplica-se por \textit{prob(H)}, denominada \textit{Prior}, é a probabilidade da hipótese ser verdadeira; por fim, o denominador \textit{prob(A)}, é a probabilidade total.
    
\begin{equation} \label{eq:teorema-bayes}
prob(H|D) = \frac{prob(D|H)prob(H)}{prob(D)}
\end{equation}

\par Apesar de parecer um teorema simples, seu alcance está na capacidade de interpretação. No caso do Naive Bayes, assume-se que os atributos são independentes -- daí a denominação de ingênuo (\textit{naive} em inglês).

\subsection{Árvores de Decisão}
\label{sec:arvores-decisao}

\subsection{Algoritmos Genéticos}
\label{sec:alg-geneticos}

% % % % % % % % % % % % % % % % % % % % % % % % % % % % % % % % % % % % % % 
% % % % % % % % % % % % TRABALHOS RELACIONADOS % % % % % % % % % % % % % % % 
% % % % % % % % % % % % % % % % % % % % % % % % % % % % % % % % % % % % % % 

\chapter{Trabalhos Relacionados}
\label{sec:trab-relacionados}

% % % % % % % % % % % % % % % % % % % % % % % % % % % % % % % % % % % % % % 
% % % % % % % % % % % % % % % % PROPOSTA % % % % % % % % % % % % % % % % % % 
% % % % % % % % % % % % % % % % % % % % % % % % % % % % % % % % % % % % % % 

\chapter{Proposta}
\label{sec:proposta}

\section{Definição dos Atributos}
\label{sec:def-atributos}

\section{Arquitetura de Extração de Tuítes}
\label{sec:arq-extracao}

\par Como mostra a Figura \ref{fig:arquitetura}.

\begin{figure}[ht]
     \caption{Arquitetura adotada para extração de tuítes}
\centering
\includegraphics[width=0.6\textwidth]{figuras/arquitetura.png}
\vspace{\baselineskip} %%% linha em branco para atender a norma
    \fonte{Autor.}
\label{fig:arquitetura}
\end{figure}

% % % % % % % % % % % % % % % % % % % % % % % % % % % % % % % % % % % % % % 
% % % % % % % % % % % % % % EXPERIMENTOS  % % % % % % % % % % % % % % % % % 
% % % % % % % % % % % % % % % % % % % % % % % % % % % % % % % % % % % % % % 

\chapter{Experimentos}
\label{sec:experimentos}

% % % % % % % % % % % % % % % % % % % % % % % % % % % % % % % % % % % % % % 
% % % % % % % % % % % % % % % CONCLUSÕES % % % % % % % % % % % % % % % % % %
% % % % % % % % % % % % % % % % % % % % % % % % % % % % % % % % % % % % % % 

\chapter{Conclusões}
\label{sec:conclusao}


% % % % % % % % % % % % % % % % % % % % % % % % % % % % % % % % % % % % % % 
% % % % % % % % % % % % FIM DAS PAGINAS TEXTUAIS % % % % % % % % % % % % % % 
% % % % % % % % % % % % % % % % % % % % % % % % % % % % % % % % % % % % % % 


% % % % % % % % % % % % % % % % % % % % % % % % % % % % % % % % % % % % % % 	
% % % % % % % % % % % % % BIBLIOGRAFIA  % % % % % % % % % % % % % % % % % % 
% % % % % % % % % % % % % % % % % % % % % % % % % % % % % % % % % % % % % % 	

\bibliografia{referencias}  %%%%% BIBLIOGRAFIA -> INCLUIR NAS CHAVES O NOME DO ARQUIVO *.BIB	
	
% % % % % % % % % % % % % % % % % % % % % % % % % % % % % % % % % % % % % 	
% % % % % % % % % % % % % APÊNDICES % % % % % % % % % % % % % % % % % % %
% % % % % % % % % % % % % % % % % % % % % % % % % % % % % % % % % % % % % 	
\apendice %%%% TEXTOS A PARIR DESTE PONTO SERÃO CONSIDERADOS APÊNDICES

\chapter{Demonstração de algo}
\label{sec:apendice-demonst-algo}
        \par Algo como apêndice.  
        
% % % % % % % % % % % % % % % % % % % % % % % % % % % % % % % % % % % % % % 	
% % % % % % % % % % % % % % % ANEXOS  % % % % % % % % % % % % % % % % % % % 
% % % % % % % % % % % % % % % % % % % % % % % % % % % % % % % % % % % % % % 	
\anexo    %%%% TEXTOS A PARIR DESTE PONTO SERÃO CONSIDERADOS ANEXOS

\chapter{Algo interessante que alguém fez}
\label{sec:anexo-algo-interessante}
         \par Algo como anexo.

\end{document}

